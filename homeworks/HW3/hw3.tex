\documentclass{article}

% Language setting
% Replace `english' with e.g. `spanish' to change the document language
\usepackage{biblatex} %Imports biblatex package
\addbibresource{sample.bib}
\usepackage[english]{babel}
\usepackage{array}
\usepackage{amsmath}
\usepackage{pythonhighlight}
\newcolumntype{P}[1]{>{\centering\arraybackslash}p{#1}}
\newcolumntype{M}[1]{>{\centering\arraybackslash}m{#1}}

% Set page size and margins
% Replace `letterpaper' with `a4paper' for UK/EU standard size
\usepackage[letterpaper,top=2cm,bottom=2cm,left=3cm,right=3cm,marginparwidth=1.75cm]{geometry}

\usepackage{amsmath}
\usepackage{graphicx}
\usepackage[colorlinks=true, allcolors=blue]{hyperref}
\usepackage{setspace}
\usepackage{booktabs}
\usepackage[T1]{fontenc}
\usepackage{longtable}
\doublespacing

\begin{document}
\begin{titlepage}

\centering
\scshape
\vspace{\baselineskip}

%
\rule{\textwidth}{1.6pt}\vspace*{-\baselineskip}\vspace*{2pt}
\rule{\textwidth}{0.4pt}

{\Huge \textbf{\textsc{NPRE 449: Homework 3 \\
\vspace{15pt}}}}

\rule{\textwidth}{0.4pt}\vspace*{-\baselineskip}\vspace{3.2pt}
\rule{\textwidth}{1.6pt}\vspace{6pt}
%%\centerline{\textit{University of Illinois at Urbana-Champaign}} 
\vspace{1.5\baselineskip}


\large \centerline{\textbf{Author:} Nathan Glaser}
\large \centerline{\textbf{Net-ID:} nglaser3}
\quad

\vfill
\large \centerline{September 18, 2024}
%
\pagenumbering{gobble}
\end{titlepage}

\tableofcontents
\newpage
\pagenumbering{arabic}

\section*{Question 1}
\addcontentsline{toc}{section}{\protect\numberline{}Question 1}

A Pressurized Water Reactor (PWR) is a variant of a Light Water Reactor (LWR). This reactor type utilizes light water (Standard $H_2O$ as opposed to $D_2O$) as both its coolant and moderator. This water is highly pressurized, typically at 15.5 MPa. This pressurization is to increase the boiling point of saturated water, reducing the amount of boiling occurring inside the reactor core itself, ensuring water is constantly in contact with the hot surfaces in the core. the reactor vessel will have between 2 and 4 pairs of pipes (for systems using U-Tube steam generators, Once-Through steam generators have 1 hot leg for two cold legs), termed hot and cold legs. Each inlet cold leg will have an outlet hot leg; both are at the same elevation with respect to the reactor core. Water enters the reactor vessel through the cold legs and through the down-comer, which forces the water into the lower-plenum below the reactor core. From here, the water is forced up through the reactor core, and out through the hot legs. Notably, the control rods of PWRs are above the reactor when not in use. The fuel of PWRs is of the form of $UO_2$ cylindrical pellets, formed into long rods. Each rod is typically clad with some rendition of Zircalloy.These rods are then placed into square assemblies with somewhere around 17x17 rods, depending on the design of the core.  Each assembly is actually an 'open' structure, meaning cross-flow between assemblies is able to occur. 

INCLUDE PICTURE

Within the primary loop there are three main components, excluding the piping. There is the reactor vessel, the steam generator, and the reactor coolant pump. The hot legs connects the reactor vessel to the steam generator; the intermediate leg connects the steam generator to the reactor coolant pump; and the cold leg connects the reactor coolant pump to the reactor vessel. The reactor vessel heats the incoming water to near-saturation temperatures. The steam generator, depending on the model, takes in the hot water from the reactor and transfers the heat to the secondary loop, creating steam. There are two standard steam generators: U-Tube and Once-Through. U-Tube steam generators involve segregating inlet hot reactor water into thousands of small tubes, and then running cold secondary loop water over these tubes; then running the heated secondary loop water through driers and separators. The maximum steam quality achieved is .9975. Once-Through steam generators are much more complex internally, having one hot leg outlet and two cold leg inlets. This steam generator type is essentially a heat exchanger, and can super heat vapor, thus achieving steam qualities greater than 1. Finally, the reactor coolant pump has a volumetric flow rate of roughly 100,000 gallons per minute, utilizing a motor with 6-10,000 horsepower. 

\section*{Question 2}
\addcontentsline{toc}{section}{\protect\numberline{}Question 2}


\section*{Question 3}
\addcontentsline{toc}{section}{\protect\numberline{}Question 3}

Thermal design limits, pertaining to PWRs and BWRs, are operational limitations on the reactors themselves due to material limitations. In nuclear systems, these thermal design limits are the limiting factor for power output, not fuel energy density. For both PWR and BWR systems, the main thermal design limitation is due to boiling water.  In both systems, other than the water, the fuel and Zircaloy cladding have thermal limits as well. For Zircaloy the cladding cannot reach 1204 $^o$C, as at this temperature the cladding oxidizes, which is a very exothermic reaction releasing 190 kJ/mol. Further, the fuel itself has a melting temperature of 2865 $^o$C, and cannot operate higher than this temperature due to changes in fissile density.  

For PWRs, the thermal design limit attributed to boiling water is called Departure from Nucleate Boiling (DNB). This occurs when water cannot physically reach the surface. This is very low-quality steam, essentially inverse of annular flow in BWRs. There is a thin steam layer directly on the surface, and immediately outside of this layer is liquid water.

For BWRs, the thermal design limit from water is called Dryout. Dryout is the point in which there is no longer any water contact with the wall. At this point the coolant is high-quality steam, that is to say that the steam contact with the surface is not simply a boundary but the entirety of the coolant. 

A thermal design margin is the ratio of the minimum parameter that causes the thermal design limit to the maximum of that parameter in your reactor system. Essentially, if you have a thermal design margin of 1.3, then the heat-flux required to reach the limit is 1.3 times, or 30\%, greater than the maximum heat-flux in your system. This is done to ensure that the thermal limitations of your system are not reached when under standard operations. Typical values for DNB are 1.3, and typical values for Dryout are 1.2-1.3. 

\section*{Question 4}
\addcontentsline{toc}{section}{\protect\numberline{}Question 4}

\section*{Question 5}
\addcontentsline{toc}{section}{\protect\numberline{}Question 5}


\end{document}