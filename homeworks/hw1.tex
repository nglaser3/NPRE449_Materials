\documentclass{article}

% Language setting
% Replace `english' with e.g. `spanish' to change the document language
\usepackage{biblatex} %Imports biblatex package
\addbibresource{sample.bib}
\usepackage[english]{babel}
\usepackage{array}
\usepackage{amsmath}
\usepackage{pythonhighlight}
\newcolumntype{P}[1]{>{\centering\arraybackslash}p{#1}}
\newcolumntype{M}[1]{>{\centering\arraybackslash}m{#1}}

% Set page size and margins
% Replace `letterpaper' with `a4paper' for UK/EU standard size
\usepackage[letterpaper,top=2cm,bottom=2cm,left=3cm,right=3cm,marginparwidth=1.75cm]{geometry}

\usepackage{amsmath}
\usepackage{graphicx}
\usepackage[colorlinks=true, allcolors=blue]{hyperref}
\usepackage{setspace}
\usepackage{booktabs}
\usepackage[T1]{fontenc}
\usepackage{longtable}
\doublespacing

\begin{document}
\begin{titlepage}

\centering
\scshape
\vspace{\baselineskip}

%
\rule{\textwidth}{1.6pt}\vspace*{-\baselineskip}\vspace*{2pt}
\rule{\textwidth}{0.4pt}

{\Huge \textbf{\textsc{NPRE 449: Homework 1 \\
\vspace{15pt}}}}

\rule{\textwidth}{0.4pt}\vspace*{-\baselineskip}\vspace{3.2pt}
\rule{\textwidth}{1.6pt}\vspace{6pt}
%%\centerline{\textit{University of Illinois at Urbana-Champaign}} 
\vspace{1.5\baselineskip}


\large \centerline{\textbf{Author:} Nathan Glaser}
\large \centerline{\textbf{Net-ID:} nglaser3}
\quad

\vfill
\large \centerline{September 1, 2024}
%
\pagenumbering{gobble}
\end{titlepage}

\tableofcontents
\newpage
\pagenumbering{arabic}


%%HERE UP
\section{Question 1}
PWR - 

BWR - 



\section{Question 2}

Decay heat in PWR is removed through two mechanisms. The first system to kick on is on the secondary side, the auxiliary feed-water system. The steps in this system coming online are:
\begin{itemize}
    \item[1.] Turbine isolation. 
    \item [2.] Pull water from condensate storage, keeping cold water in the steam generator
    \item[3.] dump steam from steam generator into condenser or vent the steam
\end{itemize}
This system comes online specifically when the reactor is first shut down. 

After some time has passed, where the decay heat reaches a certain threshold, the second system comes on and the first turns off. This second system is on the primary side, and is the residual heat remover system. This pumps water from the hot leg and passes through heat exchangers to remove heat, then pumping back into the cold leg. 

\section{Question 3}

The Emergency Core Cooling System (ECCS) has two main functions. The first is to cool the core during an accident scenario, ensuring the reactor does not experience a melt-down. The second is standard decay heat removal, occurring every time the reactor shuts down. 

There is one component of the ECCS that can operate without electrical power from either the grid or the diesel generators, this being the accumulator tank on each cold leg. This tank is filled with borated water, and is pressurized by nitrogen. If the pressure in any cold leg hits a certain minimum, all accumulator tanks release their borated water into their respective cold legs, effectively inserting a strong neutron absorber into the reactor. 

In addition to the accumulator tanks, there are multiple components within the ECCS that do not require electrical power from the grid to operate, but do rely on power from the diesel generators. Each reactor has multiple diesel generators with their only purpose being to provide power to the ECCS components. There are three separate pumps / loops available for the ECCS to utilize, depending on the conditions of the system and the flow rate requirements. The first is the high pressure injection system, which is the charging pump pumping water from the RWST. The charging pump is the pump from the chemical and volume control system (CVCS). Then, there is the intermediate pressure injection system, which performs a similar operation simply at lower pressures allowing for higher pump capacity and flow rate. Finally, there is the low pressure injection system, which is the same as the RHR system.

\section{Question 4}

\section{Question 5}


\end{document}
