\documentclass{article}

% Language setting
% Replace `english' with e.g. `spanish' to change the document language
\usepackage{biblatex} %Imports biblatex package
\addbibresource{sample.bib}
\usepackage[english]{babel}
\usepackage{array}
\usepackage{amsmath}
\usepackage{pythonhighlight}
\newcolumntype{P}[1]{>{\centering\arraybackslash}p{#1}}
\newcolumntype{M}[1]{>{\centering\arraybackslash}m{#1}}

% Set page size and margins
% Replace `letterpaper' with `a4paper' for UK/EU standard size
\usepackage[letterpaper,top=2cm,bottom=2cm,left=3cm,right=3cm,marginparwidth=1.75cm]{geometry}

\usepackage{amsmath}
\usepackage{graphicx}
\usepackage[colorlinks=true, allcolors=blue]{hyperref}
\usepackage{setspace}
\usepackage{booktabs}
\usepackage[T1]{fontenc}
\usepackage{longtable}
\doublespacing

\begin{document}
\newcommand{\newsec}[1]{
\section*{Question #1}
\addcontentsline{toc}{section}{\protect\numberline{}Question #1}
}

\newcommand{\phase}{\left(\Vec{r},t\right)}
\newcommand{\genheat}{\nabla \cdot k\phase \nabla T \phase + q'''\phase = \rho c_p \frac{\partial T \phase}{\partial t}}
\newcommand{\sscmheat}[1]{k_{#1} \nabla^2 T_{#1} \left(\Vec{r}\right) + q'''_{#1} = 0}
\newcommand{\partia}[2]{\frac{\partial #1}{\partial #2}}
\newcommand{\qppo}{\mu_a \phi_{\gamma} E}
\begin{titlepage}

\centering
\scshape
\vspace{\baselineskip}

%
\rule{\textwidth}{1.6pt}\vspace*{-\baselineskip}\vspace*{2pt}
\rule{\textwidth}{0.4pt}

{\Huge \textbf{\textsc{NPRE 449: Homework 3 \\
\vspace{15pt}}}}

\rule{\textwidth}{0.4pt}\vspace*{-\baselineskip}\vspace{3.2pt}
\rule{\textwidth}{1.6pt}\vspace{6pt}
%%\centerline{\textit{University of Illinois at Urbana-Champaign}} 
\vspace{1.5\baselineskip}


\large \centerline{\textbf{Author:} Nathan Glaser}
\large \centerline{\textbf{Net-ID:} nglaser3}
\quad

\vfill
\large \centerline{September 18, 2024}
%
\pagenumbering{gobble}
\end{titlepage}

\tableofcontents
\newpage
\pagenumbering{arabic}

\newsec{1}

To begin, the general heat diffusion equation is:
\begin{equation}
    \genheat
\end{equation}

Next, assuming constant material properties and steady state:
\begin{equation}
    \sscmheat{}
\end{equation}

tfuel:

\begin{equation}
    T_f(r) = \frac{-q'''r^2}{6k_f} - \frac{C_1}{r} + C_2
\end{equation}

BCS:

\quad symettry:

\begin{equation}
    -k_f \frac{\partial T_f(r = 0)}{\partial r} = 0
\end{equation}

or
\begin{equation}
    T(r = 0) \neq \pm \infty
\end{equation}

\quad border temp:

\begin{equation}
    T_f(r = r_o) = T_{sat,water} = 668.169 ^oF
\end{equation}

Thus we get $C_1$ to be 0 by the finiteness BC, and then $C_2$ to be:

\begin{equation}
    C_2 = 668.169 + \frac{q'''r_o^2}{6 k_f} = 957.520
\end{equation}

The maximum temp is simply the center line, or $C_2$

\newpage
\newsec{2}
To begin:
\begin{equation}
    \genheat
\end{equation}

Assuming steady state:
\begin{equation}
    \sscmheat{f}
\end{equation}

where $q'''$ is defined as:

\begin{equation}
    q''' = \qppo e^{-\mu_a x}
\end{equation}

Solving this diff-eq yields the following general solution:

\begin{equation}
    T_f(x) = -\frac{\qppo e^{-\mu_a x}}{\mu_a^2 k } + C_1x + C_2
    \label{q2gensol}
\end{equation}

Next, our boundary conditions:

\begin{equation}
    -k \partia{T}{x = 0} = h \left( T_{\infty} - T_s\right) 
\end{equation}

And 
\begin{equation}
    -k \partia{T}{x = t} = h \left(T_s - T_{\infty}\right)
\end{equation}

subsituting in \eqref{q2gensol} to these BCs yields, and rearranging:

\begin{equation}
    \frac{k}{h} \cdot C_1 - C_2 = -\frac{\qppo}{\mu_a h} - \frac{\qppo}{\mu_a^2 k} - T_{\infty}
\end{equation}

\begin{equation}
    \left( h t + k \right) \cdot C_1+ h \cdot C_2 = \frac{\qppo e^{-\mu_a t}}{\mu_a} \left( \frac{h}{\mu_a k} - 1\right) + h T_{\infty}
\end{equation}

Forming into matrices and solving:

\begin{equation}
    \begin{bmatrix}
    \frac{k}{h} & -1 \\
    h t + k & h
    \end{bmatrix} 
    \begin{bmatrix}
        C_1 \\
        C_2
    \end{bmatrix}
    =
    \begin{bmatrix}
        -\qppo\left(\frac{1}{\mu_a h} - \frac{1}{\mu_a^2 k}\right) - T_{\infty} \\
        \frac{\qppo e^{-\mu_a t}}{\mu_a} \left( \frac{h}{\mu_a k} - 1\right) + h T_{\infty}
    \end{bmatrix}
\end{equation}

\begin{equation}
    \begin{bmatrix}
        C_1 \\ C_2
    \end{bmatrix}
     = 
     \begin{bmatrix}
         -2599.532 \quad \frac{^oF}{ft}\\ 1695.497  \quad ^oF
     \end{bmatrix}
\end{equation}


\subsection{Part A}
Now, to find the surface temperatures we simply plug in $C_1$, $C_2$, and the x-location of the surfaces into \eqref{q2gensol}. 

\[
\boxed{T_f(x = 0) = 481.156} \]\[
\boxed{T_f(x = t) = 366.720}
\]

\subsection{Part B}
Next to find the maximum temperature, we simply differentiate the temperature distribution, set it equal to 0, solve for x, and then plug this x back into the temperature distribution. 
\begin{align}    
    0 = \partia{T}{x} = \frac{\qppo e^{-\mu_a x}}{\mu_ak}+C_1 \\
    \frac{-C_1 \mu_a k}{\qppo} = e^{-\mu_a x} \\
    \frac{ln\left(\frac{-C_1 \mu_a k}{\qppo}\right)}{\mu_a} = x \\
    x = 0.167
\end{align}

\[
\boxed{T_f(x = 0.167) = 912.506 ^oF}
\]

\newpage
\newsec{3}
\begin{equation}
    \genheat
\end{equation}

\newsec{4}

\newsec{5}

\end{document}